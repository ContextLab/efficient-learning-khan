\title{cover letter}
%
% See http://texblog.org/2013/11/11/latexs-alternative-letter-class-newlfm/
% and http://www.ctan.org/tex-archive/macros/latex/contrib/newlfm
% for more information.
%
\documentclass[11pt,stdletter,orderfromtodate,sigleft]{newlfm}
\usepackage{blindtext, xfrac, animate, hyperref, pxfonts, geometry}

  \setlength{\voffset}{0in}

\newlfmP{dateskipbefore=0pt}
\newlfmP{sigsize=20pt}
\newlfmP{sigskipbefore=10pt}
 
\newlfmP{Headlinewd=0pt,Footlinewd=0pt}

\newcommand{\journal}{Nature Human Behavior}
\newcommand{\articletype}{Article}
\newcommand{\myTitle}{Text embedding models yield high-resolution insights into conceptual knowledge from short multiple-choice quizzes}
\newcommand{\corresponding}{Jeremy R. Manning}


\namefrom{\vspace{-0.3in}\corresponding}
\addrfrom{
  Dartmouth College\\
  Department of Psychological and Brain Sciences\\
  HB 6207 Moore Hall\\
  Hanover, NH, 03755}
\addrto{}
\dateset{\today}

 
\greetto{To the editors of \textit{\journal}:}


 
\closeline{Sincerely,}
%\usepackage{setspace}
%\linespread{0.85}
% How will your work make others in the field think differently and move the field forward?
% How does your work relate to the current literature on the topic?
% Who do you consider to be the most relevant audience for this work?
% Have you made clear in the letter what the work has and has not achieved?

\begin{document}
\begin{newlfm}

  We have enclosed our manuscript entitled \textit{\myTitle} to be considered for
  publication as an \textit{\articletype}. In our manuscript, we develop and
  test a mathematical framework, based on natural language processing models,
  for tracking and characterizing the acquisition of real-world conceptual
  knowledge.

  We asked our participants to watch two lecture videos from the Khan Academy
  platform and to take short multiple-choice quizzes before and after each
  lecture. We used a natural language processing model to capture the
  conceptual content of each moment of the lectures, and of each question. This
  model represents each lecture as a ``trajectory'' through a high-dimensional
  embedding space, and each question as a point in the same space. We designed
  the embedding space to map conceptually related content onto nearby
  coordinates. We used these embeddings, along with participants' automatically
  graded responses to the quiz questions, to characterize nuances in
  participants' knowledge. Essentially, our approach models ``knowledge'' about
  a given concept as the weighted proportion of questions a participant answers
  correctly, where the weights reflect how much each question is ``about'' the
  given concept (according to the natural language processing model). Whereas
  the raw proportion of correctly answered questions can characterize
  content-nonspecific knowledge (e.g., something akin to ``how \textit{much}
  the participant knows''), these weighted proportions provide estimates about
  \textit{what} the participant knows.

  Our manuscript makes several important contributions. First, we define our
  framework and show how we can jointly model the content of multiple
  real-world course lectures and questions. We designed our framework to be
  scalable to a very broad range of content areas. Second, we use participants'
  behavioral data to evaluate the knowledge estimates produced by our
  framework. We show that the knowledge estimates can reliably differentiate
  whether participants will answer individual held-out quiz questions correctly
  versus incorrectly. Third, we use our framework to construct detailed
  ``maps'' of what participants know (based on their responses from a single
  quiz) and how their knowledge changes over time (based on differences between
  the knowledge maps estimated from successive quizzes). These maps demonstrate
  how knowledge (and changes in knowledge) about \textit{any} conceptual
  content expressible by the text embedding model may be estimated from the
  same small set of multiple-choice questions.
  
In addition to showing how detailed insights into what people know may be
gleaned from a small number of questions, our work also lays the foundation for
several exciting future applications and research areas. For example, the
knowledge maps produced by our framework could be used by educators to help
guide their lessons and better understand their students. The maps could also
be used as a core component of a new generation of automated tutors that adapt
to-be-presented content according to the latest estimate of what a student
knows. More broadly, we also see potential overlap with other research domains
that consider how information ``flows'' between people. In our manuscript we
focus on information flow from teachers (i.e., the course lectures) to
students. But we see analogous approaches being used to characterize other
social domains as well, such as between friends having a conversation,
doctor-patient interactions, romantic partnerships, business meetings, and
more. We suggest that the degree of overlap between two individuals' knowledge
maps, in a particular region of content space, may serve as a predictive signal
for how effectively those individuals will be able to communicate about content
in that region of content space. We expect that this article will be of
interest to a broad audience including educators, cognitive psychologists,
social psychologists, natural language processing researchers, and others.

Thank you for considering this manuscript, and we hope you will find it suitable
for publication in \textit{\journal}.


\end{newlfm}
\end{document}
